\newpage
\section*{Введение}
\addcontentsline{toc}{section}{Введение}

Драйвер устройства -- это низкоуровневая программа, содержащая специфический код для работы с устройством, которая позволяет пользовательским программам (и самой ОС) управлять им стандартным образом.

В современных версиях ядра Linux по умолчанию присутствуют все необходимые драйверы для всех поддерживаемых устройств\cite{Love}. Но для старых версий ядра иногда приходится заниматься бэк-портированием драйверов или даже написанием из с нуля, чтобы обеспечить корректную работу железа.

Все устройства можно разделить на:
\begin{itemize}
\item \textbf{Символьные}. Чтение и запись устройства идет посимвольно. Примеры таких устройств: клавиатура, последовательные порты.
\item \textbf{Блочные}. Чтение и запись устройства возможны только блоками, обычно по 512 или 1024 байта. Пример - жесткий диск.
\item \textbf{Сетевые интерфейсы}. Отличаются тем, что не отображаются на файловую систему, т.е. не имеют соответствующих файлов в директории /dev, поскольку из-за специфики этих устройств работа с сетевыми устройствами как с файлами неэффективна. Пример - сетевая карта (eth0).
\end{itemize}

В распоряжении имеется относительно старая материнская плата ASUS P5B на чипсете Intel P965, со встроенной сетевой картой на основе Realtek RTL8111B, для которой будет разработан драйвер, работающий в старой версии ядра Linux.

Это довольно популярная платформа r8169, для которой открыта спецификация. Ссылка на неё приводится в списке использованных материалов.