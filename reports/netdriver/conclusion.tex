\newpage
%------------------------------------------------
\section*{Заключение}
\addcontentsline{toc}{section}{Заключение}

В данной работе были рассмотрены теоретические и практические вопросы разработки драйверов для устройств ядра Linux.

Создание драйвера устройств для Linux является досточно простой задачей так как она сводится к написанию новой функции и определении ее в системе. Тем самым, когда доступно устройство, присущее драйверу, система вызывает вашу функцию.

Однако, необходимо помнить, что драйвер устройства является частью ядра. Это означает, что драйвер запускается на уровне ядра и обладает большими возможностями: записать в любую область памяти, изменить или даже полностью удалить данные, и даже повредить физические устройства. 

Спецификация  PCI определяет большое количество типов передачи данных и структур.  Все  алгоритмы реализованы в ядре linux, а программисту драйверов  предоставляется удобный и простой интерфейс в виде  набора  функций, макросов, структур.