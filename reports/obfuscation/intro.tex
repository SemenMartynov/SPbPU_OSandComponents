\newpage
\section*{Введение}
\addcontentsline{toc}{section}{Введение}

Обфускацией программ называется такое эквивалентное преобразование программ, которое придает программе форму, затрудняющую понимание алгоритмов и структур данных, реализуемых программой, и препятствующую извлечению из текста программы определенной секретной информации, содержащейся в ней. Поскольку обфускация программ может найти широкое применение при решении многих задач криптографии и компьютерной безопасности, задаче оценки стойкости обфускации придается очень большое значение, начиная с самых первых работ в этой области. За последние два года написано свыше 70-ти статей по этой теме, что является показателем настоящий гонки между исследовательскими группами.

Давая пользователям доступ к установочным файлам программ, компании неизбежно раскрывают свои профессиональные секреты и наработки, и ничто не останавливает конкурентов от противоправного копирования и воровства чужих алгоритмов. Обратим внимание и на другой пример, это важные обновления (патчи), исправляющие ошибки в операционных системах. Почти мгновенно очередное обновление анализируется хакерами, они выявляют проблему которую это обновление чинит, и атакуют пользователей, не успевших вовремя обновиться, пользователей.