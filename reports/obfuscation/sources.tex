\newpage
\section*{}
\addcontentsline{toc}{section}{Список литературы}

\begin{thebibliography}{00}

\bibitem{Cit1} Diffie W., Hellman M. New directions in cryptography // IEEE Transactions on
Information Theory, IT-22(6), 1976, p.644-654.

\bibitem{Cit2} Collberg C., Thomborson C., Low D. A Taxonomy of Obfuscating Transformations //
Technical Report, N 148, Univ. of Auckland, 1997.

\bibitem{Cit3}  Barak B., Goldreich O., Impagliazzo R., Rudich S., Sahai A., Vadhan S. and Yang K. «On the (im) possibility of obfuscating programs.» CRYPTO 2001.

\bibitem{Cit4}  Garg S., Gentry C., Halevi S., Raykova M., Sahai A., and Waters B. «Candidate indistinguishability obfuscation and functional encryption for all circuits.» FOCS 2013.

\bibitem{Cit5} Goldwasser S., and Guy N. R. «On best-possible obfuscation.» TCC 2007.

%\bibitem{Dushutina}Душутина Е.В. Межпроцессные взаимодействия в операционных системах. Учебное пособие -- СПб.: 2014 -- 135 стр.

\end{thebibliography}