\newpage
\section{Системный загрузчик}

В Minix используется очень простой загрузчик. Все файлы представлены ниже.

\textbf{bootblock.s (8.0K)} -- Проверка диска, а также пример загрузочного сектора.

\textbf{a.out2com (4.0K)} -- Скрипт, превращающий Minix a.out файл в com-файл.
У меня возникает такое подозрение, что a.out в Minix3 также реализован не полностью и представляет собой дополненный заголовком 32-х битный raw-файл.

\textbf{boothead.s (4.0K)} -- Поддержка BIOS для boot.c .
Файл содержит начальный и низкоуровневый код для вторичного загрузчика. Содержит функции для диска, tty (консоли) - ввода с клавиатуры, копирования памяти в произвольную точку.

\textbf{rawfs.h (4.0K)} -- В этих файлах осуществляется поддержка файловой системы Minix v1 v2
\begin{Verbatim}[frame=single]
* One function needs to be provided by the outside world:
*
* void readblock(off_t blockno, char *buf, int block_size);
* Read a block into the buffer. Outside world handles
* errors.
\end{Verbatim}

\textbf{rawfs.c (4.0K)} -- В этих файлах осуществляется поддержка файловой системы Minix v1 v2

\textbf{addaout.c (4.0K)} -- Маленькая утилита для добавления заголовка minix a.out к произвольному файлу. Это позволяет использовать произвольные данные в загрузочном образе так, что эти данные становятся образом участка оперативной памяти.

\textbf{boot.c (16K)} -- Загружает и запускает Minix.

\textbf{boot.h (4.0K)} -- Информация между различными частями процесса загрузки.

\textbf{bootimage.c (4.0K)} -- Загружает образ и запускает его.

\textbf{doshead.s (8.0K)} -- Файл содержит стартовую и низкоуровневою поддержку вторичной программы загрузчика. Данный вариант загрузчика запускается как com-файл из-под DOS-а.

\textbf{image.h (4.0K)} -- Информация между инсталляцией загрузчика и загрузчиком.

\textbf{installboot.c (4.0K)} -- Делает устройство загрузочным.

\textbf{jumpboot.s (4.0K)} -- Этот код может быть помещён в любой свободный загрузочный сектор, подобный первому сектору расширенной партиции, партицию файловой системы, отличной от базовой (root), или в основной загрузочный сектор. Этот код загружает новый загрузчик, диск партиция и слайс (субпартиция) помещается в данный код утилитой installboot. Если нажата клавиша ALT, то диск, партиция и субпартиция вводятся вручную. Ручной интерфейс используется (по умолчанию) также в том случае, если installboot (по какой-то причине) не занёс свои данные для загрузки.

\textbf{masterboot.s (4.0K)} -- Код первичного загрузочного сектора.

\textbf{mkfhead.s (4.0K)} -- Содержит начальный код и код низкоуровневой поддержки MKFILE.COM.

\textbf{mkfile.c (4.0K)} -- Код MKFILE.COM. Работает в DOS, создаёт файл, который может быть использован ОС Minix как диск.

\textbf{updateboot.sh (4.0K)} -- Скрипт устанавливает загрузочный сектор.