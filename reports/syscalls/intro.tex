\newpage
\section*{Введение}
\addcontentsline{toc}{section}{Введение}

В языке C для осуществления операций ввода-вывода или работы с памятью используются механизмы стандартной библиотеки языка, объявленные в заголовочном файле stdio.h. Все эти механизмы являются надстройками над низкоуровневыми механизмами ввода-вывода ядра операционной системы. Пользовательские программы взаимодействуют с ядром операционной системы посредством специальных механизмов, называемых системными вызовами (system calls, syscalls). Внешне системные вызовы реализованы в виде обычных функций языка C, однако каждое обращение к таким функциям приводит к передаче управления непосредственно ядру операционной системы. Список всех системных вызовов Linux можно найти в файле /usr/include/asm/unistd.h\cite{IvanovN}.

Основным предназначением ядра всякой операционной системы, является обслуживание системных вызовов из выполняющихся в системе процессов (операционная система тратит на это порядка 99\% своего времени)\cite{Cilyuric}. Возникает системный вызов когда пользовательский процесс требует некоторой службы, реализуемой ядром (такой как открытие файла), и вызывает специальную функцию (например, open). В этот момент пользовательский процесс переводится в режим ожидания. Ядро анализирует запрос, пытается его выполнить и передает результаты пользовательскому процессу, который затем возобновляет свою работу. 

Системные вызовы в общем случае защищают доступ к ресурсам, которыми управляет ядро, при этом самые большие категории системных вызовов имеют дело с вводом/выводом (open, close, read, write, poll и многие другие), процессами (fork, execve, kill и т.д.), временем (time, settimeofday и т.п.) и памятью (mmap, brk и пр.) Под это категории подпадают практически все системные вызовы\cite{Maxwell}.

При изучении реализации системого вызова может оказаться, что ожидания пользователя отличаются от имеющегося кода. Во-первых, библиотека С в Linux реализует некоторые системные вызовы полностью в терминах других системных вызовов. Например, реализация waitpid сводится к простому вызову wait4, однако в документации на обе функции ссылаются как на системные вызовы. Другие, более традиционные системные вызовы, наподобие sigmask и ftime, реализованы почти полностью в библиотеке С, а не в ядре Linux.

Системный вызов должен возвращать значение типа int и только int. В соответствие с принятыми соглашениями, возвращаемое значение равно 0 или любому положительному числу в случае успеха и любому отрицательному числу в случае неудачи. Это вполне согласуется с теми подходами к разработке, которые приняты в сообществе C-разработчиков -- в случае, когда какая-либо функция из стандартной библиотеки С завершается неудачей, она устанавливает глобальную целочисленную переменную errno для отражения природы возникшей ошибки; те же соглашения актуальны и для системных вызовов. Однако, способы, в соответствие с которыми это происходит в случае с системными вызовами, невозможно предугадать, изучая лишь исходный код ядра. В случае сбоя системные вызовы возвращают отрицательные значения кодов ошибок, а за оставшуюся обработку отвечает стандартная библиотека C. В нормальных ситуациях системные функции ядра не вызываются непосредственно из пользовательского кода, а через тонкий слой кода в рамках стандартной библиотеки С, который в точности ответственен за подобного рода трансляцию\cite{Maxwell}.

Исторически сложилось, что с некоторого момента отрицательные значения возврата из системных вызовов больше не указывают на наличие ошибки. Несколько системных вызовов (подобных lseek) реализованы таким образом, что они даже в случае успеха возвращают большие отрицательные значения; в настоящий момент возвращаемые значения, соответствующие ошибке, лежат в пределах от -1 до -4095. Теперь стандартная библиотека С более избыточна в интерпретации значений возврата из системных вызовов (ядро при получении отрицательных значений возврата не предпринимает никаких особых действий).

В данной работе мы изучим назначение некоторых системных вызовов, связанных с работой планировщика и проследим их вызов из пользовательского кода.