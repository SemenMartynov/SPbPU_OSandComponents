\newpage
\section*{Введение}
\addcontentsline{toc}{section}{Введение}

Большинство компьютерных систем могут исполнять только команды, находящиеся в оперативной памяти компьютера, в то время как современные операционные системы в большинстве случаев хранятся на жёстких дисках, загрузочных CD-ROM, USB дисках или в локальной сети.

После включения компьютера в его оперативной памяти нет операционной системы. Само по себе, без операционной системы, аппаратное обеспечение компьютера не может выполнять сложные действия, такие как, например, загрузку программы в память. Таким образом мы сталкиваемся с парадоксом: для того, чтобы загрузить операционную систему в память, мы уже должны иметь операционную систему в памяти.

Решением данного парадокса является использование специальной компьютерной программы, называемой начальным загрузчиком. Эта программа не обладает всей функциональностью операционной системы, но её достаточно для того, чтобы загрузить другую программу, которая будет загружать операционную систему. Часто используется многоуровневая загрузка, в которой несколько небольших программ вызывают друг друга до тех пор, пока одна из них не загрузит операционную систему.

Рассмотрим стандартный процесс загрузки Linux системы с момента подачи питания до получения доступа к командному интерпретатору.