\newpage
%------------------------------------------------
\section*{Заключение}
\addcontentsline{toc}{section}{Заключение}

В данной работе были рассмотрены механизмы загрузки исполняемых приложений в операционных системах семейства Windows и Linux.

Современные подходы к разработке больших приложений предполагают использование динамических библиотек, обладающих своими особенностями.

Достоинства:
\begin{itemize}
\item экономия памяти за счёт использования одной библиотеки несколькими процессами;
\item разработка различных модулей на различных языках;
\item возможность исправления ошибок (достаточно заменить файл библиотеки и перезапустить работающие программы).
\end{itemize}

Недостатки:
\begin{itemize}
\item возможность нарушения API (при внесении изменений в библиотеку существующие программы могут перестать работать);
\item конфликт версий динамических библиотек (разные программы могут ожидать разные версии библиотек);
\item доступность одинаковых функций по одинаковым адресам в разных процессах (упрощает эксплуатацию уязвимостей).
\end{itemize}

Главной особенностью динамических библиотек является ускорение процесса разработки и предоставление хорошо протестированных решений, что является важнейшими задачами в индустрии.
