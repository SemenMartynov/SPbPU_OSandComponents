\newpage
\section*{Введение}
\addcontentsline{toc}{section}{Введение}

В связи с тем, что сегодня уровень сложности программного обеспечения очень высок, разработка приложений с использованием только какого-либо языка программирования (например, языка C) значительно затрудняется. Программист должен затратить массу времени на решение стандартных задач по созданию многооконного интерфейса. Реализация технологии связывания и встраивания объектов потребует от программиста еще более сложной работы.

Чтобы облегчить работу программиста, следует пере использовать ранее написанный код. С одной стороны это позволяет не решать одну задачу дважды, с другой -- появляются дополнительные требования по оформлению существующих решений и их распространению. Долгое время в Unix (а потом Linux) среде распространение велось с исходных кодах. При этом предполагалось, что пользователь достаточно грамотен для работы с таким источником.

С распространением персональных компьютеров и приложений для них, получили популярность динамические библиотеки, которые позволяли, в частости, обновлять приложения без их пере сборки конечным пользователем, использовать различные языки для решения разных задач и даже упрощение локализации.

В данной работе рассматривается процесс загрузки приложений на операционных системах семейства Windows и Linux и порядок работы с динамическими библиотеками.