\newpage
\section*{Постановка задачи}
\addcontentsline{toc}{section}{Постановка задачи}

\vspace{2em}

В рамках данной работы необходимо написать полезную программу для ОС семейства Linux и Windows.
Программа должна быть выполнена в качестве резидентного (не демона) приложения.
Далее переписать ту же программу с использованием динамически загружаемой библиотеки.

Таким образом, в результате работы должно получиться четыре программы:
\begin{itemize}
\item Резидентное приложение для Windows собранное единым модулем
\item Резидентное приложение для Windows с динамической библиотекой (.dll)
\item Резидентное приложение для Linux собранное единым модулем
\item Резидентное приложение для Linux с динамической библиотекой (.so)
\end{itemize}

\vspace{1em}

В процессе работы требуется изучить принцип загрузки приложений в различных операционных системах и описать особенности приложений использующих динамические библиотеку.
