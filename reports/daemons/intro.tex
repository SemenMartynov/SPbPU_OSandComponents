\section*{Введение}
\addcontentsline{toc}{section}{Введение}

В каждую минуту времени компьютер выполняет множество задач. Не все эти задачи происходят в графическом режиме и пользователь может их наблюдать. Многие вещи (служба печати, времени, индексации) скрыта, но продолжает работать. Это обеспечивает возможность компьютера в одно время быть сервером печати, сервером доступа к файлам и воспроизводить музыку. Эти службы (или демоны) обладают определенными особенностями, которые будут рассмотрены в данной работе.

Вторым важным вопросом является распространение кода программ. Исходный код позволяет всегда быть уверенным в том, что приложение делает то, что обещал разработчик, но на практике подобный подход не всегда возможет как с точки зрения защиты исходного кода коммерческих продуктов, так и по причине долгой сборки (к примеру браузер Mozilla Firefox может компилироваться почти 20 часов на среднем компьютере). Эта проблема решается распространением бинарных пакетов, которые необходимо правильно организовать на компьютере пользователя.