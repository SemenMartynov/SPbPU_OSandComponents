\newpage
\section{Дистрибуция пакетов в Linux}

Программное обеспечение в ОС Ubuntu Linux распространяется в так называемых deb-пакетах. Обычно при установке программы из репозитория система автоматически скачивает и устанавливает deb-пакеты. Главной причиной использовать этот путь является автоматическое разрешение зависимостей. Программу можно установить, только если уже установлены пакеты, от которых она зависит. Такая схема позволяет избежать дублирования данных в пакетах (например, если несколько программ зависят от одной и той же библиотеки, то не придётся пихать эту библиотеку в пакет каждой программы -- она поставится один раз отдельным пакетом). В отличие от, например, Slackware или Windows, в Ubuntu зависимости разрешаются пакетным менеджером (Synaptic, apt, Центр приложений, apt-get, aptitude) -- он автоматически установит зависимости из репозитория. Зависимости придётся устанавливать вручную, если нужный репозиторий не подключен, недоступен, если нужного пакета нет в репозитории, если вы ставите пакеты без использования пакетного менеджера (используете Gdebi или dpkg), если вы устанавливаете программу не из пакета (компилируете из исходников, запускаете установочный run/sh скрипт). Операционные системы на базе Debian распространяют пакеты deb, на базе RedHat -- rpm.

\subsection{Создание DEB/RPM/TGZ пакетов}

CheckInstall -- это удобная утилита, позволяющая создавать бинарные пакеты для Linux из исходного кода приложения. После компиляции программного обеспечения checkinstall может автоматически сгенерировать Slackware-, RPM- или Debian-совместимый пакет, который впоследствии может быть полностью удалён через соответствующий менеджер пакетов. Эта возможность является предпочтительной при установке любых пакетов\cite{Cit4}.

\textbf{Установка программы checkinstall}

Установка пакета checkinstall не должна вызвать особых сложностей. В операционных системах, использующих DEB пакеты, установка производится командой:

\begin{Verbatim}[frame=single]
user@host$ sudo apt-get install checkinstall
\end{Verbatim}

В операционной системе, использующей RPM пакеты, установка пакета checkinstall выполняется командой:

\begin{Verbatim}[frame=single]
user@host$ sudo rpm -i checkinstall
\end{Verbatim}

Если такой пакет в Вашей ОС не обнаружен, то следует посетить домашнюю страницу проекта и скачать требуемую версию для Вашего дистрибутива:

\url{http://checkinstall.izto.org/download.php}

\textbf{Компилирование исходников}

Далее следует перейти в каталог с программой и провести её компиляцию.

Программа, которая была рассмотрена в предыдущем разделе может быть собрана следующим образом.

\begin{Verbatim}[frame=single]
user@host$ g++ --std=c++14 main.cpp -o netmonitor
\end{Verbatim}

\textbf{Создание DEB-пакета из исходного кода}

Программа checkinstall создает и устанавливает пакет для основных ОС. Тип пакета (DEB или RPM) checkinstall определяет сам. Для жесткого указания типа создаваемого пакета используем команду checkinstall с ключами:

Создает и устанавливает RPM пакет

\begin{Verbatim}[frame=single]
user@host$ sudo checkinstall -R
\end{Verbatim}

Создает и устанавливает DEB пакет
\begin{Verbatim}[frame=single]
user@host$ sudo checkinstall -D
\end{Verbatim}

Создает и устанавливает TGZ пакет (дистрибутивы: Slackware, Zenwalk, DeepStyle, Vektorlinux, Mops)
\begin{Verbatim}[frame=single]
user@host$ sudo checkinstall -S
\end{Verbatim}

Далее следует ответить на несколько вопросов. По умолчанию все ответы на задаваемые вопросы подходят в большинстве случаев, поэтому везде нажимаем Enter.

\subsection{Создание PKGBUILD}

Пользовательский репозиторий Arch Linux (Arch User Repository, AUR) -- это поддерживаемое сообществом хранилище ПО для пользователей Arch. Он содержит описания пакетов (файлы PKGBUILD), которые позволят скомпилировать пакет из исходников с помощью makepkg и затем установить его, используя pacman. В AUR пользователи могут добавлять свои собственные сборки пакетов (PKGBUILD и другие необходимые файлы). Сообществу предоставлена возможность голосовать за эти пакеты или против них. Если пакет становится популярным, распространяется под подходящей лицензией и может быть собран без дополнительных сложностей, то, вероятно, он будет перенесен в репозиторий community (непосредственно доступный при помощи утилит pacman и abs)\cite{Cit4}.

Файл PKGBUILD по сути напоминает Makefile, и требует установки значений следующих переменных в зависимости от пакета:

\begin{itemize}
\item pkgname -- название пакета. Можно использовать только строчные английские буквы. Значение этой переменной большой роли не играет, но может помочь, если установить сюда имя рабочей директории, или, например, имя файла с исходным кодом (*.tar.gz), который требуется загрузить
\item pkgver -- версия пакета. Эта переменная может содержать буквы, цифры, знаки препинания, но не может содержать дефисов. Содержимое этой переменной зависит от метода присвоения версий (major.minor.bugfix, major.date, и т.д.) который использует программа. Чтобы следующие шаги были наиболее эффективными и лёгкими, рекомендуется включить номер версии в имя файла с исходным кодом. 
\item pkgrel -- число, которое нужно увеличивать каждый раз после новой сборки пакета. При первой сборке пакета значение pkgrel должно быть установлено в "1". Цель этой переменной состоит в том, чтобы различать разные сборки пакета одной и той же версии.
\item pkgdesc -- краткое описание пакета, обычно не более 76 символов.
\item arch -- список архитектур, где может быть использован данный PKGBUILD (обычно это "i686"). 
\item url -- адрес веб-сайта программы, где заинтересовавшиеся могут получить более подробную информацию о программе.
\item license -- тип лицензии (может быть 'unknown').
\item depends -- список пакетов, разделенный пробелами, которые должны быть установлены до использования пакета. Во избежании проблем, имена пакетов заключаются в апострофы ('), а весь массив в скобки. Используя математическое "больше или равно", можно указать минимальную допустимую версию пакета-зависимости.
\item makedepends -- список пакетов, которые потребуются для сборки пакета, но которые не нужны для его использования.
\item provides -- список пакетов, необходимость в которых пропадает, так как собираемый пакет выполняет, по крайней мере, похожие функции.
\item conflicts -- список пакетов, которые, если установлены, могут создать проблемы во время использования собираемого пакета.
\item replaces -- список пакетов, которые заменит собираемый пакет.
\item source -- список файлов, которые потребуются во время сборки пакета. Здесь должна быть ссылка на архив с исходным кодом программы (в большинстве случаев такая ссылка представляет из себя HTTP или FTP ссылку, заключённую в кавычки).
\item md5sums -- список контрольных сумм для файлов из предыдущей переменной, разделенных пробелами и заключённых в апострофы. Как только станут доступны все файлы из списка source, md5 суммы файлов будут автоматически сгенерированы и проверены на соответствие с этим списком.
\end{itemize}