%------------------------------------------------
\section*{Заключение}
\addcontentsline{toc}{section}{Заключение}

Программы, работающие в фоновом режиме широко распространены как в Windows так и в Linux. Для создания сервисов Windows можно использовать встроенные средства, которые значительно упрощают жизнь администратора (раньше этот процесс был много сложнее), в то время как в Linux демон должен пройти определённое количество обязательных шагов (форк, переход в корень файловой системы, закрытие стандартных дескрипторов...). С точки зрения использования системного журнала для логирования программы, то в Linux это реализованно на много проще (всего три posix вызова), чем в Windows (где требуется создавать файл манифеста).

Дистрибуция программ производится одинаково легко как в Windows так и в Linux благодаря утилитам с удобным графическим/текстовым интерфейсом. Стоит отметить, что установочный пакет Windows как правило самодостаточный, в то время как в Linux широко распространена политика зависимостей, для решения которых во время установки может потребоваться доступ к интернету и репозиторию основных пакетов.