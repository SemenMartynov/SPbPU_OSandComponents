\newpage
\section*{}
\addcontentsline{toc}{section}{Список литературы}

\begin{thebibliography}{00}

\bibitem{Cit1}Лав Р. Linux. Системное программирование. 2-е изд. -- СПб.: Питер, 2014 -- 448 стр.

\bibitem{Cit2}Собел М. Г. Linux. Администрирование и системное программирование. -- СПб.:  Питер , 2011 -- 880 стр.

\bibitem{Cit3}Иванов Н. Программирование в Linux. 2-е изд. -- СПб.:  БХВ-Петербург, 2012 -- 400 стр.

\bibitem{Cit4}Уорд Б. Внутреннее устройство Linux. -- СПб.:  Питер, 2016 -- 384 стр.

\bibitem{Cit5}Петцольд Ч. Программирование для Microsoft Windows 8. -- СПб.:  Питер, 2014 -- 1008 стр.

\bibitem{Cit6}Microsoft Software Developer Network, ст. kb251192. Создание службы Windows с помощью программы Sc.exe. https://support.microsoft.com/ru-ru/kb/251192 [Дата обращения 25 апреля 2016]

\bibitem{Cit7}Харт Дж. Системное программирование в среде Windows. -- M.: 2005

%\bibitem{Dushutina}Душутина Е.В. Межпроцессные взаимодействия в операционных системах. Учебное пособие -- СПб.: 2014 -- 135 стр.

\end{thebibliography}